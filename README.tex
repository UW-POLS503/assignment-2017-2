\documentclass[]{article}
\usepackage{lmodern}
\usepackage{amssymb,amsmath}
\usepackage{ifxetex,ifluatex}
\usepackage{fixltx2e} % provides \textsubscript
\ifnum 0\ifxetex 1\fi\ifluatex 1\fi=0 % if pdftex
  \usepackage[T1]{fontenc}
  \usepackage[utf8]{inputenc}
\else % if luatex or xelatex
  \ifxetex
    \usepackage{mathspec}
  \else
    \usepackage{fontspec}
  \fi
  \defaultfontfeatures{Ligatures=TeX,Scale=MatchLowercase}
\fi
% use upquote if available, for straight quotes in verbatim environments
\IfFileExists{upquote.sty}{\usepackage{upquote}}{}
% use microtype if available
\IfFileExists{microtype.sty}{%
\usepackage{microtype}
\UseMicrotypeSet[protrusion]{basicmath} % disable protrusion for tt fonts
}{}
\usepackage[margin=1in]{geometry}
\usepackage{hyperref}
\hypersetup{unicode=true,
            pdftitle={POLS 503: Assigment 2},
            pdfborder={0 0 0},
            breaklinks=true}
\urlstyle{same}  % don't use monospace font for urls
\usepackage{color}
\usepackage{fancyvrb}
\newcommand{\VerbBar}{|}
\newcommand{\VERB}{\Verb[commandchars=\\\{\}]}
\DefineVerbatimEnvironment{Highlighting}{Verbatim}{commandchars=\\\{\}}
% Add ',fontsize=\small' for more characters per line
\usepackage{framed}
\definecolor{shadecolor}{RGB}{248,248,248}
\newenvironment{Shaded}{\begin{snugshade}}{\end{snugshade}}
\newcommand{\KeywordTok}[1]{\textcolor[rgb]{0.13,0.29,0.53}{\textbf{#1}}}
\newcommand{\DataTypeTok}[1]{\textcolor[rgb]{0.13,0.29,0.53}{#1}}
\newcommand{\DecValTok}[1]{\textcolor[rgb]{0.00,0.00,0.81}{#1}}
\newcommand{\BaseNTok}[1]{\textcolor[rgb]{0.00,0.00,0.81}{#1}}
\newcommand{\FloatTok}[1]{\textcolor[rgb]{0.00,0.00,0.81}{#1}}
\newcommand{\ConstantTok}[1]{\textcolor[rgb]{0.00,0.00,0.00}{#1}}
\newcommand{\CharTok}[1]{\textcolor[rgb]{0.31,0.60,0.02}{#1}}
\newcommand{\SpecialCharTok}[1]{\textcolor[rgb]{0.00,0.00,0.00}{#1}}
\newcommand{\StringTok}[1]{\textcolor[rgb]{0.31,0.60,0.02}{#1}}
\newcommand{\VerbatimStringTok}[1]{\textcolor[rgb]{0.31,0.60,0.02}{#1}}
\newcommand{\SpecialStringTok}[1]{\textcolor[rgb]{0.31,0.60,0.02}{#1}}
\newcommand{\ImportTok}[1]{#1}
\newcommand{\CommentTok}[1]{\textcolor[rgb]{0.56,0.35,0.01}{\textit{#1}}}
\newcommand{\DocumentationTok}[1]{\textcolor[rgb]{0.56,0.35,0.01}{\textbf{\textit{#1}}}}
\newcommand{\AnnotationTok}[1]{\textcolor[rgb]{0.56,0.35,0.01}{\textbf{\textit{#1}}}}
\newcommand{\CommentVarTok}[1]{\textcolor[rgb]{0.56,0.35,0.01}{\textbf{\textit{#1}}}}
\newcommand{\OtherTok}[1]{\textcolor[rgb]{0.56,0.35,0.01}{#1}}
\newcommand{\FunctionTok}[1]{\textcolor[rgb]{0.00,0.00,0.00}{#1}}
\newcommand{\VariableTok}[1]{\textcolor[rgb]{0.00,0.00,0.00}{#1}}
\newcommand{\ControlFlowTok}[1]{\textcolor[rgb]{0.13,0.29,0.53}{\textbf{#1}}}
\newcommand{\OperatorTok}[1]{\textcolor[rgb]{0.81,0.36,0.00}{\textbf{#1}}}
\newcommand{\BuiltInTok}[1]{#1}
\newcommand{\ExtensionTok}[1]{#1}
\newcommand{\PreprocessorTok}[1]{\textcolor[rgb]{0.56,0.35,0.01}{\textit{#1}}}
\newcommand{\AttributeTok}[1]{\textcolor[rgb]{0.77,0.63,0.00}{#1}}
\newcommand{\RegionMarkerTok}[1]{#1}
\newcommand{\InformationTok}[1]{\textcolor[rgb]{0.56,0.35,0.01}{\textbf{\textit{#1}}}}
\newcommand{\WarningTok}[1]{\textcolor[rgb]{0.56,0.35,0.01}{\textbf{\textit{#1}}}}
\newcommand{\AlertTok}[1]{\textcolor[rgb]{0.94,0.16,0.16}{#1}}
\newcommand{\ErrorTok}[1]{\textcolor[rgb]{0.64,0.00,0.00}{\textbf{#1}}}
\newcommand{\NormalTok}[1]{#1}
\usepackage{graphicx,grffile}
\makeatletter
\def\maxwidth{\ifdim\Gin@nat@width>\linewidth\linewidth\else\Gin@nat@width\fi}
\def\maxheight{\ifdim\Gin@nat@height>\textheight\textheight\else\Gin@nat@height\fi}
\makeatother
% Scale images if necessary, so that they will not overflow the page
% margins by default, and it is still possible to overwrite the defaults
% using explicit options in \includegraphics[width, height, ...]{}
\setkeys{Gin}{width=\maxwidth,height=\maxheight,keepaspectratio}
\IfFileExists{parskip.sty}{%
\usepackage{parskip}
}{% else
\setlength{\parindent}{0pt}
\setlength{\parskip}{6pt plus 2pt minus 1pt}
}
\setlength{\emergencystretch}{3em}  % prevent overfull lines
\providecommand{\tightlist}{%
  \setlength{\itemsep}{0pt}\setlength{\parskip}{0pt}}
\setcounter{secnumdepth}{0}
% Redefines (sub)paragraphs to behave more like sections
\ifx\paragraph\undefined\else
\let\oldparagraph\paragraph
\renewcommand{\paragraph}[1]{\oldparagraph{#1}\mbox{}}
\fi
\ifx\subparagraph\undefined\else
\let\oldsubparagraph\subparagraph
\renewcommand{\subparagraph}[1]{\oldsubparagraph{#1}\mbox{}}
\fi

%%% Use protect on footnotes to avoid problems with footnotes in titles
\let\rmarkdownfootnote\footnote%
\def\footnote{\protect\rmarkdownfootnote}

%%% Change title format to be more compact
\usepackage{titling}

% Create subtitle command for use in maketitle
\newcommand{\subtitle}[1]{
  \posttitle{
    \begin{center}\large#1\end{center}
    }
}

\setlength{\droptitle}{-2em}
  \title{POLS 503: Assigment 2}
  \pretitle{\vspace{\droptitle}\centering\huge}
  \posttitle{\par}
  \author{}
  \preauthor{}\postauthor{}
  \predate{\centering\large\emph}
  \postdate{\par}
  \date{2017-04-21}

\usepackage{amsmath}
\DeclareMathOperator{\sd}{sd}

\begin{document}
\maketitle

This assignment works through an example in Yule (1899):

Yule (1899) is a published example multiple regression analysis in its
modern form.\footnote{See Freedman (1997), Stigler (1990), Stigler
  (2016), and Plewis (2017) for discussions of Yule (1899).}

Yule wrote this paper to analyze the effect of policy changes and
implementation on pauperism (poor receiving benefits) in England under
the \href{https://en.wikipedia.org/wiki/English_Poor_Laws}{English Poor
Laws}. In 1834, a new poor law was passed that established a national
welfare system in England and Wales. The New Poor Law created new
administrative districts (Poor Law Unions) to adminster the law. Most
importantly, it attempted to standardize the provision of aid to the
poor. There were two types of aid provided: in-relief or aid provided to
paupers in workhouses where they resided, and out-relief or aid provided
to paupers residing at home. The New Poor Law wanted to decrease
out-relief and increase in-relief in the belief that in-relief, in
particular the quality of life in workhouses, was a deterrence to
poverty and an encouragement for the poor to work harder to avoid
poverty.

Yule identifies that there are various potential causes of the change in
rate of pauperism, including changes in the (1) law, (2) economic
conditions, (3) general social character, (4) moral character, (5) age
distribution of the population (pg. 250).

He astutely notes the following:

\begin{quote}
If, for example, we should find an increase in the proportion of
out-relief associated with (1) an increase in the proportion of the aged
to the whole population, and also (2) an increase in the rate of
pauperism, it might be legitimate to interpret the result in the sense
that changes in out-relief and pauperism were merely simultaneous
concomitants of changes in the proportion of aged-the change of
pauperism not being a direct consequence of the change of
administration, but both direct consequenices of the change in age
distribution. It is evidently most important that we should be able to
decide between two such differenit ilnterpretations of the same facts.
This the method I have used is perfectly competernt to do --- Yule (1899
pg. 250)
\end{quote}

\section{Setup}\label{setup}

\begin{Shaded}
\begin{Highlighting}[]
\KeywordTok{library}\NormalTok{(}\StringTok{"tidyverse"}\NormalTok{)}
\KeywordTok{library}\NormalTok{(}\StringTok{"modelr"}\NormalTok{)}
\end{Highlighting}
\end{Shaded}

While only a subset of the original data of Yule (1899) was printed in
the article itself, Plewis (2015) reconstructed the orginal data and
Plewis (2017) replicated the original paper. This data is included in
the package \textbf{datums}. This package is not on CRAN, but can be
downloaded from github. \textbf{IMPORTANT} install the latest version of
\textbf{datums} since a few fixes were recently made to the
\texttt{pauperism} dataset.

\begin{Shaded}
\begin{Highlighting}[]
\CommentTok{# devtools::install_github("jrnold/datums")}
\KeywordTok{library}\NormalTok{(}\StringTok{"datums"}\NormalTok{)}
\end{Highlighting}
\end{Shaded}

The data for Yule (1899) is split into two data frames:
\texttt{pauperism\_plu} contains data on the Poor Law Unions (PLU), and
\texttt{pauperism\_year}, panel data with the PLU-year as the unit of
observation.

\begin{Shaded}
\begin{Highlighting}[]
\NormalTok{pauperism <-}
\StringTok{  }\KeywordTok{left_join}\NormalTok{(datums}\OperatorTok{::}\NormalTok{pauperism_plu, datums}\OperatorTok{::}\NormalTok{pauperism_year,}
            \DataTypeTok{by =} \StringTok{"ID"}\NormalTok{)}
\end{Highlighting}
\end{Shaded}

The data consist of 599 PLUs and the years: 1871, 1881, 1891 (years in
which there was a UK census).

Yule (1899) is explcitly using regression for causal inference. The
outcome variable of interest is:

\begin{itemize}
\tightlist
\item
  \textbf{Pauperism} the percentage of the population in receipt of
  relief of any kind, less lunatics and vagrants
\end{itemize}

The treatment (policy intervention) is the ration of numbers receiving
outdoor relief to those receiving indoor relief.

\begin{itemize}
\tightlist
\item
  \textbf{Out-Relief Ratio:} the ratio of numbers relieved outdoors to
  those relieved indoors
\end{itemize}

He will control for two variables that may be associated with the
treatment

\begin{itemize}
\tightlist
\item
  \textbf{Proportion of Old:} the proportion of the aged (65 years) to
  the whole population since the old are more likely to be poor.
\item
  \textbf{Population:} in particular changes in population that may be
  proxying for changes in the economic, social, or moral factors of
  PLUs.
\end{itemize}

There is also \textbf{Grouping of Unions}, which is a locational
classification based on population density that consists of Rural,
Mixed, Urban, and Metropolitan.

Instead of taking differences or percentages, Yule worked with ``percent
ratio differences'', \(100 \times \frac{x_{t}}{x_{t-1}}\), because he
did not want to work with negative signs, presumably a concern at the
because he was doing arithmetic by hand and this would make calculations
more tedious or error-prone.

\subsection{Original Specification}\label{original-specification}

Run regressions of \texttt{pauper} using the yearly level data with the
following specifications. In Yule (1899), the reg

\begin{itemize}
\tightlist
\item
  \emph{M1:}
  \texttt{paupratiodiff\ \textasciitilde{}\ outratiodiff\ +\ year\ +\ Type}
\item
  \emph{M2:}
  \texttt{paupratiodiff\ \textasciitilde{}\ outratiodiff\ +\ (popratiodiff\ +\ oldratiodiff)\ *\ (year\ +\ Type)}
\item
  \emph{M3:}
  \texttt{-1\ \ +\ paupratiodiff\ \textasciitilde{}\ (outratiodiff\ +\ popratiodiff\ +\ oldratiodiff)\ *\ (year\ +\ Type)}
\item
  \emph{M3:}
  \texttt{paupratiodiff\ \textasciitilde{}\ (outratiodiff\ +\ popratiodiff\ +\ oldratiodiff)\ *\ (year\ +\ Type)}
\end{itemize}

\begin{enumerate}
\def\labelenumi{\arabic{enumi}.}
\item
  Present the regressions results in a regression table
\item
  Interpret the coefficients for \texttt{outratiodiff} for each model.
\item
  Write the equations for each or all models, and describe the model
  with a sentence or two. Try to be as concise as possible. Look at
  recent journal articles for examples of the wording and format.
\item
  What is the difference between \emph{M3} and \emph{M4}. What are the
  pros and cons of each parameterization?
\item
  Conduct F-tests on the hypotheses:
\item
  All interactions in \emph{M4} are 0
\item
  The coefficients on \texttt{outratiodiff} in \emph{M4} are the same
  across years
\item
  The coefficients on \texttt{outratiodiff} in \emph{M4} are the same
  across PLU Types
\item
  The coefficients on \texttt{outratiodiff} in \emph{M4} are the same
  across PLU Types and years.
\item
  What is the predicted value of the median PLU in each year and PLU
  Type for these models. Include confidence intervals. Plot these as
  point-ranges with the estimate and confidence intervals.
\item
  As previously, calculate the predicted value of the median PLU in each
  year and PLU Type. But instead of confidence intervals include the
  prediction interval. How do the confidence and prediction intervals
  differ? What are their definitions?
\end{enumerate}

\subsection{Functional Forms}\label{functional-forms}

The regression line of the model estimated in Yule (1899) (ignoring the
year and region terms and interactions) can be also written as \[
\begin{aligned}[t]
100 \times \frac{\mathtt{pauper2}_t / \mathtt{Popn2_t}}{\mathtt{pauper2}_{t-1} / \mathtt{Popn2_{t-1}}} 
&= \beta_0 + \beta_1 \times 100 \times \frac{\mathtt{outratio}_t}{\mathtt{outratio_{t-1}}} \\
& \quad + \beta_2 \times 100 \times \frac{\mathtt{Popn65}_t / \mathtt{Popn2}_{t}}{\mathtt{Popn65}_{t-1} / \mathtt{Popn2}_{t-1}} + \beta_3 \times 100 \times \frac{\mathtt{Popn2}_t}{\mathtt{Popn2}_{t - 1}}
\end{aligned}
\]

\begin{enumerate}
\def\labelenumi{\arabic{enumi}.}
\tightlist
\item
  Take the logarithm of each side, and simplify so that
  \(\log(\mathtt{pauper2}_t/\mathtt{pauper2}_{t -1})\) is the outcome
  and the predictors are all in the form
  \(\log(x_t) - \log(x_{t - 1}) = \log(x_t / x_{t - 1})\).
\item
  Estimate the model with logged difference predictors, Year, and month
  and interpret the coefficient on \(\log(outratio_t)\).
\item
  What are the pros and cons of this parameterization of the model
  relative to the one in Yule (1899)? Focus on interpretation and the
  desired goal of the inference rather than the formal tests of the
  regression. Can you think of other, better functional forms?
\end{enumerate}

\subsection{Non-differenced Model}\label{non-differenced-model}

Suppose you estimate the model (\emph{M5}) without differencing,

\begin{verbatim}
pauper2 ~ outratio + (Popn2 + Prop65) * (year + Type)
\end{verbatim}

\begin{itemize}
\tightlist
\item
  Interpret the coefficient on \texttt{outratio}. How is this different
  than model \emph{M5}
\item
  What accounts for the different in sample sizes in \emph{M5} and
  \emph{M2}?
\item
  What model do you think will generally have less biased estimates of
  the effect of out-relief on pauperism: \emph{M5} or \emph{M2}? Explain
  your reasoning.
\end{itemize}

\subsection{Substantive Effects}\label{substantive-effects}

Read Gross (2014) and McCaskey and Rainey (2015). Use the methods
described in those papers to assess the substantive effects of out-ratio
on the rate of pauperism. Use the model(s) of your choosing.

\subsection{Influential Observations and
Outliers}\label{influential-observations-and-outliers}

\subsubsection{Influential Observations for the
Regression}\label{influential-observations-for-the-regression}

For this use \emph{M2}:

\begin{enumerate}
\def\labelenumi{\arabic{enumi}.}
\tightlist
\item
  For each observation, calculate and explain the following:
\end{enumerate}

\begin{itemize}
\tightlist
\item
  hat value (\texttt{hatvalues})
\item
  standardized error (\texttt{rstandard})
\item
  studentized error (\texttt{rstudent})
\item
  Cook's distance (\texttt{cooksd})
\end{itemize}

\begin{enumerate}
\def\labelenumi{\arabic{enumi}.}
\setcounter{enumi}{1}
\tightlist
\item
  Create an outlier plot and label any outliers. See the example
  \href{https://jrnold.github.io/intro-methods-notes/outliers.html\#iver-and-soskice-data}{here}
\item
  Using the plot and rules of thumb identify outliers and influential
  observations
\end{enumerate}

\subsection{Influential Observations for a
Coefficient}\label{influential-observations-for-a-coefficient}

\begin{enumerate}
\def\labelenumi{\arabic{enumi}.}
\item
  Run \emph{M2}, deleting each observation and saving the coefficient
  for \texttt{outratiodirff}. This is a method called the jackknife. You
  can use a for loop to do this, or you can use the function
  \texttt{jackknife} in the package
  \href{https://github.com/jrnold/resamplr}{resamplr}.

  \begin{itemize}
  \tightlist
  \item
    For which observations is there the largest change in the
    coefficient on \texttt{outratiodiff}?
  \end{itemize}

  \begin{enumerate}
  \def\labelenumii{\arabic{enumii}.}
  \tightlist
  \item
    Which observations have the largest effect on the estimate of
    \texttt{outratiodiff}?
  \item
    How do these observations compare with those that had the largest
    effect on the overall regression as measured with Cook's distance?
  \item
    Compare the results of the jackknife to the \texttt{dfbeta}
    statistic for \texttt{outratiodiff}
  \end{enumerate}
\item
  Aronow and Samii (2015) note that the influence of observations in a
  regression coefficient is different than the the influence of
  regression observations in the entire regression. Calculate the
  observation weights for \texttt{outratiodiff}.

  \begin{enumerate}
  \def\labelenumii{\arabic{enumii}.}
  \tightlist
  \item
    Regress \texttt{outratiodiff} on the control variables
  \item
    The weights of the observations are those with the highest squared
    errors from this regression. Which observations have the highest
    coefficient values?
  \item
    How do the observations with the highest regression weights compare
    with those with the highest changes in the regression coefficient
    from the jackknife?
  \end{enumerate}
\end{enumerate}

\subsection{Omitted Variable Bias}\label{omitted-variable-bias}

An informal way to assess the potential impact of omitted variables on
the coeficient of the variable of interest is to coefficient variation
when covariates are added as a measure of the potential for omitted
variable bias (Oster 2016). Nunn and Wantchekon (2011) (Table 4)
calculate a simple statistic for omitted variable bias in OLS. This
statistic ``provide{[}s{]} a measure to gauge the strength of the likely
bias arising from unobservables: how much stronger selection on
unobservables, relative to selection on observables, must be to explain
away the full estimated effect.''

\begin{enumerate}
\def\labelenumi{\arabic{enumi}.}
\tightlist
\item
  Run a regression without any controls. Denote the coefficient on the
  variable of interest as \(\hat\beta_R\).
\item
  Run a regression with the full set of controls. Denote the coefficient
  on the variable of interest in this regression as \(\hat\beta_F\).
\item
  The ratio is \(\hat\beta_F / (\hat\beta_R - \hat\beta_F)\)
\end{enumerate}

Calculate this statistic for \emph{M2} and interpret it.

\subsection{Heteroskedasticity}\label{heteroskedasticity}

\begin{enumerate}
\def\labelenumi{\arabic{enumi}.}
\item
  Run \emph{M2} and \emph{M3} with a heteroskedasticity consistent
  (HAC), also called robust, standard error. How does this affect the
  standard errors on \texttt{outratio} coefficients? Use the
  \textbf{sandwich} package to add HAC standard errors (Zeileis 2004).
\item
  Model \emph{M3} is almost equivalent to running separate regressions
  on each combination of \texttt{Type} and \texttt{Year}.

  \begin{enumerate}
  \def\labelenumii{\arabic{enumii}.}
  \tightlist
  \item
    Run a regression on each subset of combination of \texttt{Type} and
    \texttt{Year}.
  \item
    How do the coefficients, standard errors, and regression standard
    errors (\(\sigma\)) differ from those of \emph{M3}.
  \item
    Compare the robust standard errors in \emph{M3} to those in the
    subset regressions. What is the relationship between
    heteroskedasticity and difference between the single regression with
    interactions (\emph{M3}) and the multiple regressions.
  \end{enumerate}
\end{enumerate}

\subsection{Weighted Regression}\label{weighted-regression}

\begin{enumerate}
\def\labelenumi{\arabic{enumi}.}
\tightlist
\item
  Run \emph{M2} and \emph{M3} as weighted regressions, weighted by the
  population (\texttt{Popn}) and interpret the coefficients on
  \texttt{outratiodiff} and interactions. Informally assess the extent
  to which the coefficients are different. Which one does it seem to
  affect more?
\item
  What are some rationales for weighting by population? See the
  discussion in Solon, Haider, and Wooldridge (2013) and Angrist and
  Pischke (2014).
\end{enumerate}

\textbf{BELOW THIS STILL IN PROGRESS}

\subsection{Average Marginal Effects}\label{average-marginal-effects}

\subsection{Cross-Validation}\label{cross-validation}

When using regression causal estimation, model specification and choice
should largely be based on avoiding omitted variables. Another criteria
for selecting models is to use their fit to the data. But a model's fit
to data should not be assessed using only the in-sample data. That leads
to overfitting---and the best model would always be to include an
indicator variable for every observation Instead, a model's fit to data
can be assessed by using its out-of-sample fit. One way to estimate the
\emph{expected} fit of a model to \emph{new} data is cross-validation.

\subsection{Bootstrapping}\label{bootstrapping}

Estimate the 95\% confidence intervals of model with simple
non-parametric bootstrapped standard errors. The non-parametric
bootstrap works as follows:

Let \(\hat\theta\) be the estimate of a statistic. To calculate
bootstrapped standard errors and confidence intervals use the following
procedure.

For samples \(b = 1, ..., B\).

\begin{enumerate}
\def\labelenumi{\arabic{enumi}.}
\tightlist
\item
  Draw a sample with replacement from the data
\item
  Estimate the statistic of interest and call it \(\theta_b^*\).
\end{enumerate}

Let \(\theta^* = \{\theta_1^*, \dots, \theta_B^*\}\) be the set of
bootstrapped statistics.

\begin{itemize}
\item
  standard error: \(\hat\theta\) is \(\sd(\theta^*)\).
\item
  confidence interval:

  \begin{itemize}
  \tightlist
  \item
    normal approximation. This calculates the confidence interval as
    usual but uses the bootstrapped standard error instead of the
    classical OLS standard error:
    \(\hat\theta \pm t_{\alpha/2,df} \cdot \sd(\theta^*)\)
  \item
    quantiles: A 95\% confidence interval uses the 2.5\% and 97.5\%
    quantiles of \(\theta^*\) for its upper and lower bounds.
  \end{itemize}
\end{itemize}

\subsection*{References}\label{references}
\addcontentsline{toc}{subsection}{References}

\hypertarget{refs}{}
\hypertarget{ref-AngristPischke2014a}{}
Angrist, Joshua D., and Jörn-Steffen Pischke. 2014. \emph{Mastering
`Metrics}. Princeton UP.

\hypertarget{ref-AronowSamii2015a}{}
Aronow, Peter M., and Cyrus Samii. 2015. ``Does Regression Produce
Representative Estimates of Causal Effects?'' \emph{American Journal of
Political Science} 60 (1). Wiley-Blackwell: 250--67.
doi:\href{https://doi.org/10.1111/ajps.12185}{10.1111/ajps.12185}.

\hypertarget{ref-Freedman_1997}{}
Freedman, David. 1997. ``From Association to Causation via Regression.''
\emph{Advances in Applied Mathematics} 18 (1). Elsevier BV: 59--110.
doi:\href{https://doi.org/10.1006/aama.1996.0501}{10.1006/aama.1996.0501}.

\hypertarget{ref-Gross2014a}{}
Gross, Justin H. 2014. ``Testing What Matters (If You Must Test at All):
A Context-Driven Approach to Substantive and Statistical Significance.''
\emph{American Journal of Political Science} 59 (3). Wiley-Blackwell:
775--88.
doi:\href{https://doi.org/10.1111/ajps.12149}{10.1111/ajps.12149}.

\hypertarget{ref-McCaskeyRainey2015a}{}
McCaskey, Kelly, and Carlisle Rainey. 2015. ``Substantive Importance and
the Veil of Statistical Significance.'' \emph{Statistics, Politics and
Policy} 6 (1-2). Walter de Gruyter GmbH.
doi:\href{https://doi.org/10.1515/spp-2015-0001}{10.1515/spp-2015-0001}.

\hypertarget{ref-NunnWantchekon2011a}{}
Nunn, Nathan, and Leonard Wantchekon. 2011. ``The Slave Trade and the
Origins of Mistrust in Africa.'' \emph{American Economic Review} 101
(7): 3221--52.
doi:\href{https://doi.org/10.1257/aer.101.7.3221}{10.1257/aer.101.7.3221}.

\hypertarget{ref-Oster2016a}{}
Oster, Emily. 2016. ``Unobservable Selection and Coefficient Stability:
Theory and Evidence.'' \emph{Journal of Business \& Economic
Statistics}, September. Informa UK Limited, 0--0.
doi:\href{https://doi.org/10.1080/07350015.2016.1227711}{10.1080/07350015.2016.1227711}.

\hypertarget{ref-Plewis2015a}{}
Plewis, Ian. 2015. ``Census and Poor Law Union Data, 1871-1891.'' SN
7822. UK Data Service; data collection.
doi:\href{https://doi.org/10.5255/UKDA-SN-7822-1}{10.5255/UKDA-SN-7822-1}.

\hypertarget{ref-Plewis2017a}{}
---------. 2017. ``Multiple Regression, Longitudinal Data and Welfare in
the 19th Century: Reflections on Yule (1899).'' \emph{Journal of the
Royal Statistical Society: Series A (Statistics in Society)}, February.
Wiley-Blackwell.
doi:\href{https://doi.org/10.1111/rssa.12272}{10.1111/rssa.12272}.

\hypertarget{ref-SolonHaiderWooldridge2013a}{}
Solon, Gary, Steven Haider, and Jeffrey Wooldridge. 2013. ``What Are We
Weighting for?'' National Bureau of Economic Research.
doi:\href{https://doi.org/10.3386/w18859}{10.3386/w18859}.

\hypertarget{ref-Stigler1990a}{}
Stigler, Stephen M. 1990. \emph{The History of Statistics: The
Measurement of Uncertainty Before 1900}. HARVARD UNIV PR.
\url{http://www.ebook.de/de/product/3239165/stephen_m_stigler_the_history_of_statistics_the_measurement_of_uncertainty_before_1900.html}.

\hypertarget{ref-Stigler2016a}{}
---------. 2016. \emph{The Seven Pillars of Statistical Wisdom}. Harvard
University Press.
\url{http://www.ebook.de/de/product/25237216/stephen_m_stigler_the_seven_pillars_of_statistical_wisdom.html}.

\hypertarget{ref-Yule1899a}{}
Yule, G. Udny. 1899. ``An Investigation into the Causes of Changes in
Pauperism in England, Chiefly During the Last Two Intercensal Decades
(Part I.).'' \emph{Journal of the Royal Statistical Society} 62 (2).
JSTOR: 249. doi:\href{https://doi.org/10.2307/2979889}{10.2307/2979889}.

\hypertarget{ref-Zeileis2004a}{}
Zeileis, Achim. 2004. ``Econometric Computing with Hc and Hac Covariance
Matrix Estimators.'' \emph{Journal of Statistical Software} 11 (1):
1--17.
doi:\href{https://doi.org/10.18637/jss.v011.i10}{10.18637/jss.v011.i10}.


\end{document}
